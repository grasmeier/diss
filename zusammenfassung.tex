\vspace{-3cm}
\noindent
\begin{center}
{\large \sffamily \textbf{Zusammenfassung}}
\end{center}
\noindent
\vspace{0.5cm}
%selectlanguage{ngerman}%Babel umschalten

%\noindent
%\begin{center}
%{\large \sffamily \textbf{Zusammenfassung}}
%\end{center}
%\noindent
%...
%
Greifen ist eine der grundlegenden F\"ahigkeiten der Manipulation. Beherrschen von Greifen ist die Voraussetzung und Garantie f\"ur den Roboter, um komplexe Manipulation Aufgaben durchzuf\"uhren. Obwohl das Robotergreifen schon \"uber einen langen Zeitraum untersucht wurde, bleibt es noch eine gro{\ss}e L\"ucke zu der menschlichen Greiff\"ahigkeit. Gerade im unstrukturierten Umfeld steht der Roboter vor vielen Herausforderungen. Erstens, die Absicht von dem Greifen hat in der Regel einen Zusammenhang mit einer Aufgabe. Eine Griffkonfiguration beschr\"ankt, was f\"ur eine Aufgabe durchgef\"uhrt werden kann. Das Ausw\"ahlen einer geeigneten Griffkonfiguration ist eine Herausforderung. Au{\ss}erdem, es existieren einer Vielzahl von Objekten in der Welt. Es ist unm\"oglich, einzelne Objekten vorher zu modellieren. Wie man Objekte behandelt, die den Robotern zum ersten Mal pr\"asentiert werden, ist eine gro{\ss}e Herausforderung. Dar\"uber hinaus, die f\"ur dem Greifen zu verwendeten Sensoren sind imperfekt und haben meistens einen Messfehler. Der Umgang mit der Messunsicherheit eines Wahrnehmungssystems f\"ur das Greifen ist ebenfalls eine Herausforderung. Die Menschen k\"onnen fragen, warum Menschen in der Lage sind, die F\"ahigkeiten des Greifens in ihrem fr\"uhen Alter zu beherrschen, w\"ahrend das Greifen eine schwierige Aufgabe f\"ur den Roboter ist. Der Grund ist, dass das Greifen eine interdisziplin\"are Aufgabe ist. Die erfordert verschiedenen Aufgaben gemeinsam zu l\"osen, darunter fallen die Wahrnehmung von der Objekten, die Schlussfolgerung von m\"oglicher Greifstrategien sowie die robuste Steuerung und \"uberwachung von Greifprozesses.

In dieser Dissertation wird ein integriertes Greifsystem vorgeschlagen. Das System betrachtet das Greifen als einen dynamischen Vorgang aus dem Zustand, von dem ein Objekt zum Zustand wahrgenommen wird, zu dem das Objekt erfa{\ss}t wird. Ein mobiler Manipulator, der das vorgeschlagene System verwendet, ist in der Lage, ein robustes Greifen unter verschiedenen Stufen von vorherigen Kenntnissen und Bedingungen durchzuf\"uhren. Die vorherigen Kenntnisse und Bedingungen bestimmen, was ist das gr\"o{\ss}te Problem, in einem bestimmten Greif-Szenario gel\"ost werden muss. Wir schlagen vor, die Differenz verschiedener Greifszenarien unter einem einheitlichen Modell darzustellen, das durch das `Bayesian Network' definiert wird. Auf diese Weise wird das Greifproblem auf eine probabilistische Weise formuliert. Das Modell definiert eine gemeinsame Verteilung der greifenden relevanten Faktoren, w\"ahrend die Unsicherheit dieser Faktoren durch probabilistische Verteilungen quantifiziert wird. Drei anspruchsvolle Erfassungsszenarien unter individuellen Bedingungen werden durch das vorgeschlagene Modell formuliert. Die Differenz des Szenarios wird durch Konditionierung der jeweiligen Faktoren im Modell unterschieden.

Im ersten Szenario wird das Greifen im Rahmen von Montageaufgaben behandelt. Das Hauptproblem besteht darin, Griffkonfigurationen auszuw\"ahlen, die die Wahrscheinlichkeit der Erf\"ullung der Aufgabenbeschr\"ankung maximieren. Ein Skill Framework wird vorgeschlagen, um eine komplette Sequenz f\"ur einen mobilen Manipulator zu erstellen, um eine Baugruppe, von Objekt-Wahrnehmung, Argumentation der Griff-Strategie zur Steuerung von Roboter-Bewegungen. Das Framework bietet eine intuitive M\"oglichkeit f\"ur einen nicht-Experten Benutzer, ein Montageproblem zu parametrisieren. Innerhalb des Frameworks werden hochqualifizierte Skills vorgeschlagen, um Pick-and-Place-Aktionen zum \"andern der Orientierung von Objekten sowie Insertionsaktionen zum Zusammenbauen von zwei Objekten durchzuf\"uhren. Die erfolgreiche Ausf\"uhrung beider Aufgaben erfordert, dass Objekte in einer bevorzugten Aufgabenorientierung erfasst werden sollen. Abweichend von fr\"uheren Arbeiten, die nur die Greiferorientierung ber\"ucksichtigen, ist unser Verfahren in der Lage, die kompletten Roboterkonfigurationen zu erzeugen, um die Anforderung zu erf\"ullen, so dass sowohl die gesamte Fahrstrecke als auch die Ausf\"uhrungszeit des Roboters reduziert werden. Dar\"uber hinaus ist die Arm-Plattform-Koordination auch im Rahmen modelliert, um die Effizienz der Ausf\"uhrung der Aufgabe weiter zu erh\"ohen.

Im zweiten Szenario wird das Greifen unbekannter Objekte adressiert. Das Hauptproblem besteht darin, eine Griffkonfiguration zu synthetisieren, die die Wahrscheinlichkeit des Kraftschlusses maximiert. Dazu werden eine neue probabilistische Objektdarstellung und ein f\"ur die Darstellung ma{\ss}geschneidertes Sensorfusionsverfahren vorgeschlagen, um die Form von Objekten vor dem Greifen zu rekonstruieren. Diese Methode eignet sich besonders zur Modellierung von Objekten mit unregelm\"a{\ss}igen Formen. Im Gegensatz zu fr\"uheren Arbeiten mit anderen Darstellungen ist es uns m\"oglich, die Rekonstruktionsunsicherheit basierend auf dem Rauschmodell von Sensoren zu modellieren. Auf diese Weise hat unsere Repr\"asentation eine wirkliche Interpretation der Modellierungsunsicherheit und kann dazu verwendet werden, Greifpunkte auf unbestimmten Fl\"achen zu bestrafen. Basierend auf der Darstellung wird ein Griffsyntheseverfahren auf der Grundlage eines simulierten Gl\"uhens vorgeschlagen, um eine gute Griffkonfiguration zu suchen. Im Vergleich zu modernster Methode, die eine \"ahnliche Darstellung verwendet, arbeitet unsere Methode auf 3D-Objekten und l\"auft um ein Vielfaches schneller.

Im dritten Szenario wird ein Sensor mit gro{\ss}er Unsicherheit verwendet, um die Greifleistung zu studieren. Die gro{\ss}e Herausforderung liegt, wie man ein robustes Greifen durchf\"uhrt, selbst wenn  die Wahrnehmungsunsicherheit ist gro{\ss}. Anders als bei bisherigen Ans\"atzen, wird eine adaptive Regelnarchitektur vorgeschlagen. Es erm\"oglicht einem Roboter, das Greifen in einer geschlossenen Schleife durchzuf\"uhren. Unter Verwendung der vorgeschlagenen Architektur wird das Wahrnehmungsergebnis dem Griffsynthesemodul durchgehend zugef\"uhrt, in dem aktuelle Griffkonfigurationen erzeugt werden. Die Bewegungsanpassung an die neue Griffkonfiguration wird durch DMP erreicht, so dass w\"ahrend der Ann\"aherungsphase die Unsicherheit der Wahrnehmung aktiv reduziert wird. Aktoren eines mobilen Manipulators k\"onnen flexibel kombiniert und in unterschiedlicher Phase eines Greifvorgangs gezielt eingesetzt werden.

Die Wirksamkeit des vorgeschlagenen integrierten Systems wird durch eine gro{\ss}e Anzahl von Experimenten der realen Welt \"uberpr\"uft. Die Vorteile der in jedem Kapitel vorgestellten Methoden werden durch Vergleich mit modernsten Methoden demonstriert.


\selectlanguage{american} %Babel zur\"ucksetzen
