\chapter*{Notations}

\manualmark
\markboth{Notations}{Notations}

\section*{Abbreviations}

\begin{longtable}[l]{ll}
APRB			& Amplitude modulated pseudo random binary\\
ATP				& Adenosine triphosphate\\
CNS				& Central nervous system\\
DIP				& Distal interphalangeal\\
DOF				& Degree of freedom\\		
EDC				& Extensor digitorum communis\\
EIP				& Extensor indices proprius\\
EM				& Error model \\
EMG				& Electromyography \\
EvMA			& Estimation of voluntary muscle activity \\
FDP				& Flexor digitorum profundus\\
FDS				& Flexor digitorum superficialis\\
GN 				& Gauss-Newton\\
GS 				& Gradient search\\
ITAE 			& Integral time multiplied absolute error\\
LM 				& Levenberg-Marquardt\\
LMS 			& Least mean squares\\
LP 				& Linearly parameterized\\
l.p.e.		& Linear persistent excitation/linearly persistently exciting\\
l.o.l.		& Location of lesion\\
LTI				& Linear time invariant \\
LAP				& Linear adaptive prediction \\
m.				& Musculus\\
MA				& Moving average\\
MAP				& Muscle action potential\\
MCP				& Metacarpophalangeal\\
MIL				& Matrix inversion lemma\\
MLP				& Multi layer perceptron\\
MU				& Motor unit\\
MVC				& Maximum voluntary contraction\\
NARX      & Nonlinear autoregressive with exogenous input\\
NFIR			& Nonlinear finite impulse response\\
NLP				& Nonlinearly parameterized\\
NMSE			& Normalized mean square error\\
NRBF			& Normalized radial basis function\\
n.l.p.e.  & Nonlinear local persistent excitation\\
PCA				& Principle component analysis\\
PID				& Proportional-integral-derivative\\
PIP				& Proximal interphalangeal\\
PSD				& Power spectral density\\
RBF				& Radial basis function\\
RLS				& Recursive least squares\\
RMS       & Root mean square\\
SNLP			& Separable nonlinearly parameterized\\
t.s.l.		& Time since lesion\\
wRMS			& Weighted RMS\\
WSS				& Wide sense stationary\\
ZDVC			& Zero degree voluntary contraction\\
\end{longtable}

\vspace{0.5cm}
\section*{Conventions}

\minisec{Scalars, Vectors, and Matrices} 

\emph{Scalars} are denoted by upper and lower case letters in italic
type. \emph{Vectors} are denoted by underlined lower case letters in italic type, as the vector $\ux$ is composed of elements $x_i$. \emph{Matrices} are denoted by underlined upper case letters in italic type, as the matrix $\uM$ is composed of elements $M_{ij}$ ($i^{\text{th}}$ row, $j^{\text{th}}$ column).

\begin{longtable}[l]{ll}
$x$ or $X$ 				& Scalar\\
$\ux$ 						& Vector \\
$\uX$ 						& Matrix\\
$\uX^{T}$ 				& Transposed of $\uX$\\
$\uX^{-1}$ 				& Inverse of $\uX$\\
$\uX^{+}$ 				& Pseudoinverse of $\uX$\\
$f(\cdot)$ 				& Scalar function \\
$\uf(\cdot)$ 			& Vector function \\
$\hat x$ 					& Estimated or predicted value of $x$ \\
$\tx$ 						& Estimation error: $\tx=x-\hx$ \\
$\overline x$			& Average value of $x$\\
$\|\cdot \|_p$ 		& p-norm \\
$\nabla f(\ux)=\frac{\partial f(\ux)}{\partial \ux}$ & Gradient vector \\
\end{longtable}



%\pagebreak
%\section*{Subscripts and Superscripts}
%
%\begin{tabular}{ll}
%
%$x^{d}$ & desired value of $x$, set value for the control loop \\
%$x_{\text{max}}$ & maximum value of $x$ \\
%$x_{\text{min}}$ & minimum value of $x$ \\
%$(\cdot )^{-1}$ & inverse \\
%$(\cdot )^+$ & pseudo-inverse\\
%$(\cdot )^T$ & transposed \\
%$(\cdot )^*$ & optimal or expected value \\
%$(\cdot )_{pos}$ & position \\
%$(\cdot )_{pose}$ & pose \\
%$(\cdot )_{tran}$ & translation \\
%$(\cdot )_{rot}$ & rotation \\
%$(\cdot )_{mono}$ & mono-focal \\
%$(\cdot )_{multi}$ & multi-focal \\
%\end{tabular}
%\nopagebreak
\vspace{0.5cm}
\section*{Symbols}

\minisec{General} 
\begin{longtable}[l]{ll}
$\uA_w$										& System matrix of the LTI-system $\mW$\\
$\ub_w$										& Input vector of the LTI-system $\mW$\\
$\uc_w$										& Output vector of the LTI-system $\mW$\\
$f_k$											& Discrete frequency variable\\
$f_{rep}$								  & Repetition rate of \rpms\\
$f_s$										  & Sampling rate\\
$G_w(s)$									& Transfer function of the LTI-system $\mW$\\
$G_{w,obs}(s)$						& Transfer function of the Luenberger observer of the LTI-system $\mW$\\
$\uI$											& Unity matrix\\
$I_c$										  & 95\,\% confidence interval\\
$j$					    					& Discrete time variable occurring in discrete integrals\\
$k$												& Discrete time variable\\
$k_{rep}$								  & Discrete repetition period of \rpms\\
$N$   										& Some discrete time period\\
$t$   										& Time\\
$T$   										& A certain time period\\
$T_1$											& Time constant of a PT$_1$-system\\
$T_p$										  & Pulse width of \rpms\\
$T_s$       							& Sampling time, $T_s=0.001\,$s throughout the thesis\\
$T_{set}$									& Settling time of a dynamic system\\
$T_{sys}$									& System time constant of a desired polynomial\\
$u$         							& Input of an single input system\\
$\uv$       							& Input vector of a multiple input system\\
$\uw$											& Unit vector\\
$\mW$											& SISO LTI-system with $\uA_w$, $\ub_w$, $\uc^T_w$\\
$\mW_{obs}$								& LTI-system constituted by a Luenberger observer\\
$\ux$      		  					& State of a dynamic system\\
$y$         							& System output\\
$\delta(k)$							  & Discrete Dirac Delta function\\
$\sigma$								  & Standard deviation\\
\end{longtable}

\minisec{System Identification} 
\begin{longtable}[l]{ll}
$A_j(u)$											& Activation function of an NRBF-network\\
$e$														& Output error \\
$e_n$													& Output error of the nonlinear subsystem $n(\cdot)$\\
$e_e$													& Augmented error\\
$e_{obs}$											& Observer error\\ 
$\uf^{\htheta}$					  		& Update term of the SLS-algorithm\\
$E(\cdot)$										& Error criterion\\
$g$														& Gain ration of modified LM-algorithm\\
$\ug(\cdot)$									& Nonlinear function that defines the adaptive law\\
$\uh$	  											& Truncated impulse response\\
$\uH$													& Hessian matrix\\
$I$						  							& Integral\\
$k_{d,c}$											& Discrete delay time of hardware and physiological delay\\
$k_h$													& Discrete time horizon of exponential forgetting\\
$k_h$													& Discrete hold time of an APRB-signal\\
$\ull$           							& Gain vector of a Luenberger Observer\\
$L(\cdot)$ 										& Model of the error criterion $E(\cdot)$\\
$m$ 													& Number of linear parameters\\
$m_r$													& Number of orthonormal basis functions\\
$m_{N1}$								  		& Number of radial basis functions of the nonlinearity $\hN_1(\alpha)$\\
$m_{N2}$											& Number of radial basis functions of the nonlinearity $\hN_2(\da)$\\
$n(\cdot)$  									& Nonlinear Subsystem\\	
$p$ 													& Total number of parameters\\
$\uP$													& Symmetric positive definite matrix\\
$q$ 													& Number of nonlinear parameters\\
$r_a$													& Actual reduction of the error criterion\\
$r_p$													& Predicted reduction of the error criterion\\
$\uQ$													& Symmetric positive definite matrix\\
$\ur_i$												& Orthonormal basis function\\
$\uR$													& Matrix of orthonormal basis functions\\
$\uR$													& Direction matrix of the LM-algorithm (damped Hessian)\\
$s_r$													& Condition ratio of the modified LM-algorithm\\
$S_i$													& Singular Values of the direction matrix $\uR$\\ 
$T_{d,c}$											& Complete delay consisting of hardware delay and physiological delay\\
$T_h$   											& Time horizon of exponential forgetting\\
$T_h$   											& Hold time of an APRB-signal\\
$V(\cdot)$										& Lyapunov candidate\\
$y_n$       									& Output of a nonlinear subsystem\\
$z$         									& Measurement noise\\

%%%%%%Griechische Buchstaben%%%%%
$\beta$												& Threshold value of the LM-algorithm\\
$\gamma$											& Estimator gain\\
$\gamma_{\heta}$							& Estimator gain for nonlinear parameters (SLS-algorithm)\\
$\gamma_{\htheta}$						& Estimator gain for linear parameters (SLS-algorithm)\\
$\delta$        							& Damping factor of the LM-algorithm\\
$\delta_s$      					    & Damping factor of the modified LM-algorithm\\
$\epsilon$								    & Output error of the SLS-algorithm\\
$\epsilon_{obs}$							& Observer error in EM C2\\
$\eta$      									& Nonlinear system parameter\\
$\theta$   							  		& Linear system parameter\\
$\kappa$											& Gain factor of the LM-algorithm\\
$\lambda$											& Forgetting factor of exponential forgetting\\
$\nu$													& Threshold value of the modified LM-algorithm\\
$\xi$       									& General system parameter \\
$\Delta \xi$					    		& Distance between two activation functions\\
$\uPi$												& Covariance matrix of the RLS-algorithm\\
$\osigma$							    		& Normalized smoothing parameter of an NRBF-network\\
$\tau$												& Time variable occurring in integrals\\
$\uphi$ 	      							& Input regressor\\
$\ouphi$											& Filtered input regressor\\
$\uPhi$												& Matrix of input regressors\\
$\uchi$     									& State of an LTI-system\\
$\upsi$												& Gradient vector\\
%$\kappa$,$\epsilon$,$\epsilon_1$ & Positive constants\\
\end{longtable}

\minisec{Neuromuscular and Biomechanical Modeling} 
\begin{longtable}[l]{ll}
%%%%%Lateinische Buchstaben%%%%%
$d_s$										& Specific density\\
$D_{rel}$								& Damping constant of the relaxation model\\
$E_{rel}$								& Elasticity constant of the relaxation model\\
$f_l$										& Scaling of the force-length-curve\\
$f_v$										& Scaling of the force-velocity-curve\\
$F_M$										& Muscle force\\
$F_s$										& Sensor force\\
$G_a(s)$								& Transfer function of the muscle twitch model\\
$G_e(s)$								& Transfer function of the temporal summation model\\
$h$											& Tendon leverage\\
$I(k)$									& Stimulation intensity\\
$I_{sat}$								& Saturation intensity of the recruitment model\\
$I_{thr}$								& Threshold intensity of the recruitment model\\
$J$											& Moment of inertia\\
$l$											& Tendon length\\
$L_i$										& Length of the i$^{\text{th}}$ finger phalanx\\
$m_i$										& Mass of the i$^{\text{th}}$ finger phalanx\\
$N_1(\alpha)$						& Static nonlinearity of the segment dynamics\\
$N_2(\da)$						& Static nonlinearity of the segment dynamics\\
$r$											& Radius of the MCP-joint\\
$R_i$										& Radius of the i$^{\text{th}}$ finger phalanx\\
$s(\alpha,\da)$				  & Spastic joint torque\\
$s_t(\alpha)$						& Tonic component of the spastic joint torque\\
$s_{ph}(\da)$						& Phasic component of the spastic joint torque\\
$T_a$										& Time constant of the muscle twitch model\\
$T_e$										& Time constant of the temporal summation model\\
$T_{d,hw}$							& Hardware delay\\
$T_{d,ph}$							& Physiological delay\\
$T_{rel}$								& Time constant of the relaxation model\\
$u(k)$									& Input of the plant "\rpmsh-stimulated muscle"\\
%%%%%Griechische  Buchstaben%%%%%
$\alpha_i$							& Angle of the i$^{\text{th}}$ joint\\
$\alpha_{sat}$					& Curvature parameter of the recruitment model\\
$\alpha_{thr}$					& Curvature parameter of the recruitment model\\
$\beta_1$, $\beta_2$		& Gain and offset parameters of the recruitment model\\ 
$\gamma_{p}$						& Pennation angle\\
$\rho(\cdot)$						& Function of motor unit recruitment\\
$\tau_{ep}$							& Elastic joint torque of a muscle-tendon unit\\
$\tau_f$								& Friction torque\\
$\tau_g$								& Gravitational torque\\
$\tau_i$								& Torque of the i$^{\text{th}}$ joint\\
$\tau_n$								& Net joint torque\\
$\tau_{me}$							& Sensor Torque of the Fingertester\\
$\tau_M$								& Muscle torque\\
$\tau_{rel}$						& Torque of relaxation characteristics\\
$\tau_{vm}$							& Torque of viscous muscle-tendon properties\\
\end{longtable}

\minisec{Applications} 
\begin{longtable}[l]{ll}
%%%%%Lateinische Buchstaben%%%%%
$a(k)$								& Measure for the voluntary muscle activity\\
$a_i$								  & EMG-amplitude\\
$b_j$									& Coefficient of the LAP-filter\\
$c(k)$								& Raw EMG-signal\\
$d_t$									& Relative difference of the tonic spasticity component evaluation\\
$d_{ph}$							& Relative difference of the phasic spasticity component evaluation\\
$K_a$									& Gain of the adaptive trajectory generation\\
$K_{RMS}$							& Length of the RMS-filter\\
$s(k)$								& EMG-signal inside the EvMA-cascade\\
$S_1$, $S_2$					& Muting periods of the EMG-amplifier\\
$w$									  & Input of a system with state feedback\\
$y_d$									& Desired output of a controlled system\\

%%%%%Griechische  Buchstaben%%%%%
$\uGamma$							& Eigenvector matrix\\
$\iota$								& Additive noise term\\
$\ukappa$					    & State feedback vector\\
$\kappa_I$					  & Gain of the integral controller\\
$\lambda_i$						& Eigenvalues of the covariance matrix\\
$\uLambda$						& Diagonal matrix of eigenvalues\\
$\nu$					        & Virtual system input\\
$\uXi$								& Transformation matrix of PCA\\
$\chi_I$					    & State variable of the integral controller\\
$\upsilon$						& Weighting factor of the wRMS-filter\\
$\uPsi$								& Covariance matrix\\
$\omega_0$						& Characteristic frequency of an ITAE-polynomial\\
\end{longtable}

\automark[section]{chapter}

