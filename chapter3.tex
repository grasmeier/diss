\chapter{A Probabilistic Approach to Grasp Synthesis}

\section{Introduction}

\section{Related work}

\section{Contributions}

\section{Probabilistic modelling of grasping success}
Grasp motion, is the key phase
in the entire grasping process. During this phase, the robot has to estimate the object pose and geometry, predict the likelihood of success with a pre-touch configuration and choose the action that maximizes the
grasping success probability. This sections gives the mathematic model of grasping success, followed by a the proof of the model with an 1-D grasping example. 
\subsection{Mathematic Model}
First we define a random variable $S \in \{ \text{success} , \text{failure} \}$ as the outcome of a specific grasp motion $\mathcal{G} = \{g_1, \dots ,g_n$\}, with $g_1, \dots ,g_n$ representing a set of gripper configurations and poses. We define $\text{P}({S = \text{success}}|\mathcal{G},\mathcal{Z})$ as the marginal probability that the grasp $\mathcal{G}$ succeeds based on the sensor measurement history $\mathcal{Z}=\lbrace z_0, \dots ,z_M \rbrace$. It is not straightforward to compute the marginal probability $\text{P}({S = \text{success}}|\mathcal{G},\mathcal{Z})$ directly without knowing the state of the object 
$x_o$. However, we can compute it by separating it into two probability terms: 
\begin{equation}
\text{P}(S) := \text{P}(S | \mathcal{G} ,\mathcal{Z}) = \int_{x_o} \text{P} (S | x_o,\mathcal{G} )\cdot p(x_o|\mathcal{Z}) dx_o. 
\label{e_grasp_success}
\end{equation}
We call the first term $P(S | x_o,\mathcal{G})$ conditional grasping success model. It indicates how likely a specific grasping action $\mathcal{G}$ will lead to a success given the object state $x_o$. The second term $p(x_o|\mathcal{Z})$ represents the posterior probability of the object given a series of observations, which quantifies the perception uncertainty. In this way, the problem of predicting the marginal grasp success probability is simplified to model the $P(S | x_o,\mathcal{G})$, and to model the perception uncertainty. 

After the marginal grasp probability $\text{P}(S)$ is computed, the next step is to find an optimal grasping motion $\mathcal{G}_\text{opt}$ that maximizes the marginal grasp success probability:
\begin{equation}
\mathcal{G}_\text{opt} = \argmax_\mathcal{G} \text{P}(S).
\label{eq_g_opt}
\end{equation}
 



\subsection{Proof of concept}

\section{Conditional grasp success model for 3-D rigid objects (CGM)}

\section{Modelling perception uncertainty}

\subsection{Model-based probability distribution estimation using extended Kalman filter}


\subsection{Model-free probability distribution estimation based on probabilistic fusion}


\section{Searching for feasible grasp configuration}

\subsection{Reducing configuration space of grippers using synergy }

\subsection{Contact surface query}

\subsection{A two-stage method for grasp synthesis}







